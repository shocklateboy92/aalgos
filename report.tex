\documentclass[a4paper,10pt]{article}
\usepackage[utf8]{inputenc}

%opening
\title{K-Linearizability in Contiguous Memory}
\author{Lasath Fernando}

\begin{document}

\maketitle

\begin{abstract}

\end{abstract}

\section{Background}
\subsection{Linearizability}
%% Redo sentence
When reasoning about concurrent objects, i.e. objects that are accessed by multiple threads simultaneously, existing correctness conditions are insufficient.

- consistency models in distributed systems programming
- correctness conditions from them
  - like Linearizability
  - one of the more strict ones

- assumes operation 'happens' at a point in time
  - other operations ordered relative to that point
  - if there exists an order of operations, to explain results, it is correct.

- math definition

\section{Related Work}
K Linearizability
\section{Algorithm}
\section{Proofs}
\subsection{Lock-Freedom}
\subsection{K-Linearizability}
\section{Results}

\end{document}
