%-----------------------------------------------------------------------------
%
%               Template for sigplanconf LaTeX Class
%
% Name:         sigplanconf-template.tex
%
% Purpose:      A template for sigplanconf.cls, which is a LaTeX 2e class
%               file for SIGPLAN conference proceedings.
%
% Guide:        Refer to "Author's Guide to the ACM SIGPLAN Class,"
%               sigplanconf-guide.pdf
%
% Author:       Paul C. Anagnostopoulos
%               Windfall Software
%               978 371-2316
%               paul@windfall.com
%
% Created:      15 February 2005
%
%-----------------------------------------------------------------------------


\documentclass{sigplanconf}

% The following \documentclass options may be useful:

% preprint      Remove this option only once the paper is in final form.
% 10pt          To set in 10-point type instead of 9-point.
% 11pt          To set in 11-point type instead of 9-point.
% authoryear    To obtain author/year citation style instead of numeric.

\usepackage{amsmath}
\usepackage{algorithmic}
\usepackage{algorithm}
\usepackage{hyperref}

\begin{document}

\special{papersize=8.5in,11in}
\setlength{\pdfpageheight}{\paperheight}
\setlength{\pdfpagewidth}{\paperwidth}

\conferenceinfo{CONF 'yy}{Month d--d, 20yy, City, ST, Country}
\copyrightyear{20yy}
\copyrightdata{978-1-nnnn-nnnn-n/yy/mm}
\doi{nnnnnnn.nnnnnnn}

% Uncomment one of the following two, if you are not going for the
% traditional copyright transfer agreement.

%\exclusivelicense                % ACM gets exclusive license to publish,
                                  % you retain copyright

%\permissiontopublish             % ACM gets nonexclusive license to publish
                                  % (paid open-access papers,
                                  % short abstracts)

\titlebanner{banner above paper title}        % These are ignored unless
\preprintfooter{short description of paper}   % 'preprint' option specified.

\title{K-linearizability in Contiguous Memory Data-Structures}
% \subtitle{Subtitle Text, if any}

\authorinfo{Lasath Fernando}
           {University of New South Wales}
           {lyfe227@cse.unsw.edu.au}
% \authorinfo{Name2\and Name3}
%            {Affiliation2/3}
%            {Email2/3}

\maketitle

\newcommand{\op}{\texttt}

\begin{abstract}
K-linearizability has proven to be an effective way to increase the performance
of data structures in highly concurrent environments by relaxing their
semantic guarantees. While they have already been used extensively
in queues and stacks, here we present a method of applying it to data-structures
that require contiguous memory, such as a vector. We demonstrate
its effectiveness by relaxing the \op{push\_back} operation to write
to one of k-spots from the tail. We show how an algorithm can
be derived using prime number theory for all threads to search the
entire k-region in a unique order. Finally, we benchmark this
(feature-equivalent) vector against the vector from intel's
industry leading thread building blocks library and show it is significantly more scalable.
\end{abstract}

\category{CR-number}{subcategory}{third-level}

% general terms are not compulsory anymore,
% you may leave them out
\terms
term1, term2

\keywords
keyword1, keyword2

\section{Introduction}

Amdhal's law states that the maximum speedup of a concurrent algorithm is bounded by the portion of it that cannot be parallelized. Thus, intuitively, it follows that increasing parallelism is crucial to achieving maximum performance as the number of cores available on a computer increases. Increasing parallelism on shared data structures while maintaining correctness (linearizability) has thus far proven to be a highly non-trivial task.

Fortunately, it turns out that strict linearizability isn't actually necessary for many applications. Consider for example, a web server that keeps incoming requests in a queue so that it's worker threads can service them later. In this situation, the queue is only used to avoid starvation, and thus does not actually need to be strict FIFO: even if some operations return in a different order (by a bounded amont), fariness can still be ensured. This deviation from linearizability by reordering up to some constant $k$ sequential operations is called k-linearizability.

K-linearizability has already been used to increase the throughput and scalability of FIFO queues by turning them into k-FIFO queues[2]; LIFO stacks to k-LIFO stacks, and so on. The process for turning these data structures to thier k-linearizable equivalents has been straightforward: instantiate multiple copies of a linearizable implementation and distribute operations between individual instances with the use of a load balancer[4]. This approach has a drawback that it cannot be used for data structures whose semantics dictate that its elements be located in a contiguous block of memory, such as a vector.

Here we present a concurrent vector that is lock-free with a scalable, high throughput push\_back operation. This is the only k-linearizabile operation in the vector and it inserts an element anywhere within the bounds of the \emph{k-region}. The k-region begins at the \emph{hard-tail} (before which all elements are guaranteed to be contiguous), and is exactly $k$ elements long. It also supports random access (read, write), reserve and size operations. Note that it does not support pop\_back even though it is considered standard for a vector for performance reasons, as explained in section \autoref{sec:popback}.


This paper presents the following contributions:
\begin{enumerate}
\item A method of increasing parallelism significantly in a vector push-back
operation, and thus delivering much more performance in multi-core
machines.
\item The first algorithm that can achieve k-linearizability while storing
its elements in contiguous memory.
\item A new algorithm for reducing contention by distributing the load effectively
between adjacent elements of a vector.
\end{enumerate}

\section{Background}


\subsection{Progress Guarantees}


\subsection{Linearizability}

There are several consistency models used in distributed systems to reason about correctness of shared data access. These include, in order of strictness: eventual consistency \cite{cm-eventual}, release consistency \cite{cm-release}, causal consistency \cite{cm-causal}, sequential consistency \cite{cm-sequential}, and linearizability.

While the rest are unintuitive and difficult to use in concurrent programming, linearizability has been successfully used as a correctness condition \cite{classic-linearizability}. Intuitively, a linearizable operation is considered to 'happen' at a single point in time (called its linearization point) between its start and end, after which all operations that interact with it will behave as though it has concluded. For example, a linearization point could be successfully acquiring a lock.

More formally an operation is said to be linearizable if for every concurrent history $H$, there exists an equivalent sequential history $S$, where $S$ respects the semantics of the operation. For example, it would be invalid for the following sequence of operations to occur in a stack
$$push(A)\ push(B)\ pop(A)$$
as a stack must follow \emph{last-in-first-out}.
A concurrent history is an ordered set of operation invocations (start) and responses (end) performed by each thread. A sequential history is a history where each operation invocation (except possibly the last) is directly followed by its response.
$H$ and $S$ are equivalent if for each thread $t$ in $H$ and $S$, $H \vert t = S \vert t$ where $A\vert t$ is the sub-history of events in history $A$ for some thread $t$.


\subsection{Synchronisation Primitives}
CAS$_1$ (compare-and-swap)
Atomic read/write

\section{Related Work}

\subsection{Quasi-Linearizability}
K-linearizability, or quasi linearizability is a correctness condition
that reduces the synchronization required for concurrent data structures
by relaxing the ordering. An execution is said to be k-linearizable
if it can be made equivalent to a linearizable execution by reordering
$k-1$ operations, for $k>0$ \cite{opodis10,zhang13}.

There are currently many linearizable implementations of various \emph{last-in-first-out} (stacks) and \emph{first-in-first-out} (queues) data structures. Since the operational semantics of these data structures provide no constraints on the locality or layout of their data in memory, the following approach can be used:
\begin{itemize}
 \item take an existing linearizable implementation
 \item create $k$ instances
 \item use a load balancer to distribute operations between instances
\end{itemize}

This is effectively a simplified version of the approach taken by \citet{kfifo1} for their k-FIFO queue. Achieving k-linearizability in data-structures that impose restrictions on memory layout -- such as a vector, for example -- require modifying operations on it, as we will see in section \ref{sec:implementation}.

\subsection{Lock-free Vector}
Since the contiguous memory data-structure we're using to show the effects of k-linearizability is a vector, we based off the lock-free vector implementation by \citet{dechev06}. However, the two key differences from it is that our \op{push\_back} operation will be k-linearizable, and we will not include a \op{pop\_back} operation.

Our decision to not include a \op{pop\_back} operation is due to 2 key reasons
\begin{enumerate}
  \item Implementing a high throughput concurrent \op{push\_back} and \op{pop\_back} combination and showing that their interactions are linearizable is an extremely non-trivial task.
  \item Since our purpose is to demonstrate how $k$-linearizability can be used to improve performance, it is sufficient to implement one of them.
\end{enumerate}


\section{Algorithms}
\label{sec:implementation}

\subsection{Layout}
A traditional vector consists of a single block of dynamically allocated memory, containing all elements in memory, where an index is an offset from the start of the block. This grows with each insertion, until all indices are filled. Once the block is completely filled, a new block is allocated (usually of twice the size) and all existing elements are copied to the new one. Since all values are stored continuously in a single block of memory, this usually provides good cache locality and random access is very efficient, as it requires a single add and dereference. The drawback of this approach is that the growth operation is very expensive, and could cause a significant performance impact if done too frequently.

This model however, is insufficient for highly concurrent vectors, as parallelising this growth operation is a highly non trivial task. We avoid this problem by instead using a 2 level array, like \citet{dechev06} and [TBB concurrent vector]. As such, our vector consists of a block of pointers, each pointing to blocks of doubling size, containing the values. This allows us to grow the vector by allocating and adding new blocks in a lock free manner, while keeping most of the advantages of contiguous storage, like efficient random access, good cache locality.

However, it has some limitations \cite{dechev06}.
% limitation - requires two lookups (and some bit shifts) per random access
% requires memory overhead of bucket-array
% advantage - growing vector much easier (shown in code reserve(), line blah).
% advantage outweighs the costs

\subsection{K-region}
The \ref{alg:advancetail}
- range of indices
- continuously moving
  - via \op{advance\_tail}, either manually or as part of push back
- k region is the valid area for a push-back operation to insert to
- it sits at the end
- sparsely populated
- spans several buckets

\subsubsection{Updating the tail}
% NOTE: at the point of linearization, the thread-local tail is at most k spots away from the global tail

The tail marks the start of the k-region, and the point before which
all data is packed (without sentinel values). Since all search sequences
start from the current tail, the closer it is to the actual end of
data, the faster a valid sentinel value would be found. However, since
determining the new tail requires a linear search from the previous
tail to the first sentinel value, it should be used sparingly.

We found it sufficient for each thread to attempt to do so once every
$k$ operations, when the sequence repeats itself (i.e. offset becomes
0 again).

\subsection{Memory Management}
* need thread safe memory allocator
* used tab::cache\_aligned\_allocator

\subsection{Distinguishing Filled and Empty Indices}

In a vector, an index is filled if it contains a value and empty otherwise. That is, if a read operation with that index will succeed, then it is filled. In a traditional vector, the tail index marks the last filled index. Thus, if a given index was filled or empty could be determined simply by comparing it with the tail index.

However, the k-region is a sparsely populated data structure, with no tail index. Thus, it is impossible to determine if a given index is filled or empty using said index alone. The solution, is to use the value for the index, rather than the index itself. If the value is equal to a pre-designated sentinel value, then that index is empty. If not, it is filled.

The exact value of the sentinel should be determined by the user, to ensure it is outside the domain of valid inputs. Since our implementation is implicitly shared and stores references to values, a NULL value seemed a suitable candidate.
% maybe give some examples of good sentinels
% easy to choose for pointer types - NULL
% no global solution for other cases, thus must be chosen case by case

\subsection{Bounds Checking}

\subsection{push\_back Operation}

This is the operation that benefits the most from the relaxed semantics of k-linearizability, and diverges the most from a traditional vector. Several unique issues had to be addressed relating to it.

\subsubsection{Searching for a free space}

% NOTE: This should probably get it's own section, if I end up benchmarking the effectiveness of the various searching (load-balancing) algorithms.

Since the semantics of the operation specify an entire range of locations
for the given data to be inserted to, each invocation must determine an absolute index to insert into.

The naive approach to this would be an algorithm that searches linearly
from the start of the k-region (tail) till the first sentinel value
it hits. While this would certainly work, it wouldn't leverage the
relaxed semantics and would still cause significant contention as
all concurrent push operations would try to write to the same location.

A better approach might be use a pseudo random algorithm with a fairly
even distribution to provide the offsets from the tail. Depending
on the algorithm and seed used, this may perform very well in certain
conditions, however it would be difficult to verify. It would be much
easier if we could guarantee that each thread would search the k-region
in a unique order.

To create a search pattern for each thread, we derived an algorithm
using prime number theory. Each thread $t$ is pre-assigned a unique
prime number $P_{t}$. It starts its search cycle at the tail, and
increments by $P_{t}$ until it finds a sentinel value indicating
an empty spot. i.e.
$$o_{i}=(o_{i-1}+P_{t})\bmod{k}$$
where $o_{i}$ is offset from the tail for the $i$th location to
search. Since any two prime numbers are relative primes, it can be
shown that this will always generate a unique sequence of locations
to search.

\subsection{Random Access}

One of the primary advantages of storing elements contiguously in memory is the ability to perform random access operations efficiently. In this case, we define \emph{efficiently} to be $O(1)$, conforming to the ISO C++ standard[cite-stl-ref-here], and \emph{random access operations} to be \op{read} and \op{write} specifically.

\subsubsection{Semantics}
However, since we're building a k-linearizabile data structure, the semantics provided in the STL reference for a vector are insufficient. Previous exploration of k-linearizability was mostly for k-FIFO and k-LIFO data structures which did not support random access operations; As such, a new set of semantics are required. The challenge is to define a set of semantics that are suitable in a k-linearizabile context, yet are still useful to the programmer.

There are two cases to consider: \op{read} and \op{write} operations for elements outside the k-region, and those within. For those outside, the data structure is strictly linearizabile, and conforms to the ISO C++ vector specification. For operations within, since an index by itself can be somewhat ambiguous in
a sparsely packed data structure, several disambiguation approaches were evaluated.

The first, and easiest approach would be to not support random access
reads or writes into the k-region at all. The only way to access the
last k-elements would be to finish all insertions, and perform a \op{repack}
operation which would reorder the elements in the k-region so they
are contiguous and move the tail. This effectively advances the k-region such that those elements are no longer in it, allowing them to be accessed via non k-region random access operations.

While this would certainly be the easiest, it would be a detriment
to performance as it cannot be easily parallelized. It would also
prohibit iterating through the vector while push\_back operations
are occurring. The second approach was to leave the repack operation,
but also allow iterating through the kvector. This could be achieved
by implementing an iterator that simply skipped over any sentinel
values it encountered.

This would have the observable effect of elements moving relative to each other over multiple iterations. The result of this is that the same index can refer to different elements during different operations which modify the k-region. To overcome this limitation, a write operation could take an additional parameter: expected value. The operation would now resemble a \op{compare-and-swap} primitive, allowing programmers to ensure their index refers to the correct value when updating it.

The final approach is to treat the k-region similar to a traditional vector. By either making it valid for a \op{read} to return a sentinel value, or making a \op{read} fail if a sentinel is encountered. The former is equivalent to artificially injecting values into the vector, while the latter can cause a read to fail despite appearing to be within the bounds of the vector. We opted for the latter, as it seemed to provide the best compromise between the advantages and limitations of other approaches.

\subsubsection{Implementation}

With the two level layout of data in memory that we opted for, the bucket number $b$ that an element $e$ lies in can be shown from the formula:
  $$ b = \log_2\left(\frac{e}{k}+1\right)-1 $$

The index $i$ of the element $e$ within said bucket $b$ can be shown by the formula:
  $$ i = e - 2^{b}k - k $$

Since the stored data elements can only be as large as a word, both read and write operations can be made atomic simply by using the appropriate hardware primitives.

\subsection{Pseudocode}

\newcommand{\as}{$\leftarrow$ }
\newcommand{\eq}{$=$ }

\begin{algorithm}
\caption{push\_back $thread, vector, element$}
\begin{algorithmic}
\STATE tail \as vector.tail
\REPEAT
  \STATE offset \as thread.previous\_offset
  \WHILE{at(tail + offset) \eq sentinel\_value}
    \STATE offset \as (offset + thread.prime) $\bmod{K}$
    \STATE thread.previous\_offset = offset
    \IF{offset \eq 0}
      \STATE advance\_tail()
    \ENDIF
  \ENDWHILE

  reserve(tail + offset)
\UNTIL CAS(at(tail + offset), sentinel\_value, element)
\end{algorithmic}
\label{alg:pushback}
\end{algorithm}

\begin{algorithm}
\caption{advance\_tail $vector$}
\begin{algorithmic}
\STATE old\_tail \as vector.tail
\STATE new\_tail \as old\_tail
\WHILE{at(new\_tail) $\neq$ sentinel\_value}
  \STATE new\_tail++
\ENDWHILE
\STATE CAS(vector.tail, old\_tail, new\_tail)
\end{algorithmic}
\label{alg:advancetail}
\end{algorithm}

\begin{algorithm}
\caption{reserve $vector, min_size$}
\begin{algorithmic}
\IF{vector.data[bucket\_for(min\_size)] $\neq$ NULL}
  \STATE tentative\_bucket \as alloc(bucket\_size\_for(min\_size))
  \IF{\NOT CAS(\&vector.data[bucket\_for(min\_size)], NULL, tentative\_bucket)}
    \STATE free(tentative\_bucket)
  \ENDIF
\ENDIF
\end{algorithmic}
\label{alg:reserve}
\end{algorithm}

\begin{algorithm}
\caption{read $vector, index$}
\begin{algorithmic}
\STATE value \as at(index)
\RETURN \{value \eq sentinel\_value, value\}
\end{algorithmic}
\end{algorithm}

\begin{algorithm}
\caption{write $vector, index, data$}
\begin{algorithmic}
\STATE at(index) \as data
\end{algorithmic}
\end{algorithm}




\section{Correctness and Progress Guarantees}
\subsection{Lock Freedom}
A data structure is lock-free if and only if some operation completes
after a finite number of steps system-wide have been executed on the
structure [Fraiser'02]. Thus it suffices to show that an operation
always completes, or the fact it doesn't implies that another operation does.

Read and write operations have two cases to consider: within the k-region,
and outside. Both reads and writes for elements outside the k-region are
atomic, and thus trivially wait free. In the other case, they perform a static and finite amount of reads (and a CAS operation), and so is guaranteed to terminate.

The remaining operations to consider are \emph{push\_back}, \emph{advance\_tail} and \emph{reserve}.
The \emph{advance\_tail} operation will either complete after reading a finite number of non-sentinel values, or it will live-lock because it indefinitely reads non-sentinel values, which can only occur if other operations are successfully inserting non-sentinel values. For the \emph{reserve} operation, the only interaction with global state is through a single CAS operation, whose failure indicates that another inovcation has already linearized.

The \emph{push\_back} can only be live-locked when it there are no sentinel values in the k-region, or non-sentinel values are continuously being inserted as it searches. In the latter situation, other threads are completing their operations, and in the former, the k-region will be updated within at most $k$ attempts.

\subsection{K-linearizability}

\subsubsection{K-Sequential Specification}

- what is sequential specification
\
H is a hitory of Object
\
S is a legal histroy of Object, D is the distance
$$ KSS = \left\{ H \middle\vert \exists S : D\left(H, S\right) \le k \right\} $$

For our purposes, k-linearizability means that push\_back operation

In order for \emph{push\_back} to be k-linearizabile, it is sufficient to show that all elements are inserted at most k indicies away from an equivalent linearizabile push\_back at it's linearization point.

The reset of the operations are strictly linearizabile, and as such $<$INSERT NORMAL PROOF HERE$>$.

%% While somewhat sketchy, this lays out our original logic, with some placeholders
After n pushes, $r(i) = \emptyset$. After $n+m$ pushes, $r(i) = x_j, j \in [n, n+m-1]$. After rearranging, $x_j$ is the $i$'th thing added, and before $x_j$ is added, $r(i)=\emptyset$ and after $x_j$ is added $r(i)=x_j$.

%% more concrete reasoning
Consider index $i$. After reading $i-k+1$ element, $r(i) = \emptyset$. After performing $i + k$ \op{push\_back} operations, $r(i) \ne \emptyset$, and so $r(i)$ is one of the $2k-1$ elements added. Say that $r(i)=x_j$.

In the sequential history, reorder $x_j$ so that it is the $i$th element added. Since $x_j$ is the between the $i-k+1$ and $i+k$ elements, we must move $x_j$ at most $2k-1$ positions. After re-arrangement, $r(i) = \emptyset$ before $p(x_j)$ , and $r(i) = x_j$ after and so it is a valid sequential vector history. This implies $VDist \le 2k-1$.

%% To find the highest index that's guaranteed to be empty after n pushes
Use one element to end the k-region. Then we have a gap of $k-1$ elements. And then we use the remaining $n-1$ elements, which fill indicies $[0, n-1)$. Then we have $k-1$ gaps, so we `fill in` indicies $[0, n+k-2)$ and so once we include the last element, that is $[0, n-1+(k-1)+1)$. So the index of the first guaranteed empty has index $n+k-2$.

%% To find the last element we can guarantee is full
When adding n elements,
use $k-1$ elements to 'fill' the k-region.
The we have a gap of $1$.
So we have $n-k+1$ elements left.
So those fill indicies $[0, n-k+1)$.
Therefore, $n-k$ is the last index guaranteed to contain a value.

Then work backwards. Once you have a constraint on what you want full/empty and solve for the values.

\section{Experimental Evaluation}

\section{Exploration of k-linearizabile designs with contiguous memory}
\subsection{Bounded buffer using this}

\section{Conclusion and Further Pursuit}

\appendix
\section{Appendix Title}

This is the text of the appendix, if you need one.

\acks

Acknowledgments, if needed.

% We recommend abbrvnat bibliography style.

\bibliographystyle{abbrvnat}
\bibliography{report}

% % The bibliography should be embedded for final submission.
%
% \begin{thebibliography}{}
% \softraggedright
%
% \bibitem[Smith et~al.(2009)Smith, Jones]{smith02}
% P. Q. Smith, and X. Y. Jones. ...reference text...
%
% \end{thebibliography}


\end{document}

%                       Revision History
%                       -------- -------
%  Date         Person  Ver.    Change
%  ----         ------  ----    ------

%  2013.06.29   TU      0.1--4  comments on permission/copyright notices

